\documentclass[a4paper, oneside]{book}

\usepackage{amsmath, systeme, amsfonts}

\DeclareMathOperator*{\argmax}{arg\,max}
\DeclareMathOperator*{\argmin}{arg\,min}

% -- Настройка боковых заметок ------------------------------------------------
\usepackage[a4paper, marginparwidth=6cm, marginparsep=0.8cm]{geometry}
\geometry{left=1.5cm}
\geometry{right=8cm}

\setlength{\marginparpush}{15pt}


\usepackage[T2A]{fontenc} % Use 8-bit encoding that has 256 glyphs
\usepackage[utf8]{inputenc} % Required for including letters with accents
\usepackage[russian]{babel}
\usepackage{graphicx}

% marginnote создает НЕ нумерованные плавающие боковые заметки
\usepackage{marginnote}

% Здесь делаем маленький шрифт для marginnote
\let\oldmarginpar\marginpar
\renewcommand\marginpar[1]{\-\oldmarginpar[\raggedleft\footnotesize #1]%
{\raggedright\footnotesize #1}}

% sidenotes создает НУМЕРОВАННЫЕ плавающие боковые заметки и боковые картинки
\usepackage{sidenotes}
% Здесь делаем маленький шрифт для sidenotes
\makeatletter
\RenewDocumentCommand\sidenotetext{ o o +m }{%      
    \IfNoValueOrEmptyTF{#1}{%
        \@sidenotes@placemarginal{#2}{\textsuperscript{\thesidenote}{}~\footnotesize#3}%
        \refstepcounter{sidenote}%
    }{%
        \@sidenotes@placemarginal{#2}{\textsuperscript{#1}~#3}%
    }%
}
\makeatother
% -----------------------------------------------------------------------------

% -- Настройка библиографии ---------------------------------------------------
\usepackage[backend=biber]{biblatex}
\addbibresource{biblio.bib}
% -----------------------------------------------------------------------------

% % -- Настройка форматирование глав и секций -----------------------------------
% \usepackage{titlesec}
% 
% \titleformat
% {\chapter} % command
% [display] % shape
% {\bfseries\Large\itshape} % format
% {Story No. \ \thechapter} % label
% {0.5ex} % sep
% {
%     \rule{\textwidth}{1pt}
%     \vspace{1ex}
%     \centering
% } % before-code
% [
% \vspace{-0.5ex}%
% \rule{\textwidth}{0.3pt}
% ] % after-code
% 
% 
% \titleformat{\section}[wrap]
% {\normalfont\bfseries}
% {\thesection.}{0.5em}{}
% 
% \titlespacing{\section}{12pc}{1.5ex plus .1ex minus .2ex}{1pc}
% % -----------------------------------------------------------------------------

\usepackage[breaklinks,hidelinks]{hyperref}


\title{Конечные структуры \\ {\Huge в примерах}} % The article title

\author{Садыков Нурлан}

% Абстракт
% Заметки по решению кубика рубика с помощью теории групп.

\begin{document}

\maketitle  
\chapter{Кубик Рубика}
\section{Где здесь группа?}


\begin{marginfigure}
    \includegraphics[width=1.1\columnwidth]{../pics/rubik_single_move.pdf}
    \caption{Движение желтой грани $y$}
    \label{fig:move_y}
    %\setfloatalignment{b}
\end{marginfigure}


Обозначим вращения граней кубика\marginpar{
    Под вращением грани понимаем ее поворот на $90^\circ$ по часовой стрелке
    относительно остальной части кубика.
} в соответствии с
цветом центрального стикера на грани. Это удобно потому, что вращения граней
всегда оставляют центральные стикеры на месте. Будем обозначать эти вращения
цветами соответствующих граней: $o$-оранжевый, $b$-синий, $r$-красный,
$y$-желтый, $w$-белый, $g$-зеленый. Вращения граней кубика рубика являются
образующими свободной группы. Любое слово в этой группе можно применить как
инструкцию к кубику рубика. Будем считать слово тривиальным, если после его
применения к собранному кубику, мы снова получим собранный
кубик.\marginpar{Любые промежуточные состояния применения слова к кубику не
обязаны давать собранный вариант.} Например, очевидно, что $b^4 = bbbb = e$ то
есть вращение относительно синей грани четыре раза --- тривиальное слово,
поскольку снова дает правильную сборку.

Для определенности будем считать, что мы применяем слова к кубику слева
направо. То есть слово $oby$ это сперва повернуть оранжевую грань на
$90^\circ$, потом синюю и только потом желтую.

\section{Кодировка состояния кубика}
\begin{marginfigure}
    \includegraphics[width=1.1\columnwidth]{../pics/rubik_index_involute.pdf}
    \caption{Индексация стикеров на развертке}
    \label{fig:indexes}
    %\setfloatalignment{b}
\end{marginfigure}

Каждое действие переставляет стикеры на кубике Рубика, значит группа действий
на кубике может быть описана как подгруппа группы перестановок. Чтобы явно
определить перестановку по какому-либо состоянию нужно ввести
кодировку.\marginpar{\emph{Пример неполной кодировки.} Если цветам приписать
    такие векторы $o=(0,\,0,\,1)$, $b=(1,\,0,\,0)$, $y=(0,\,1,\,0)$,
    $w=(0,\,-1,0)$, $g=(-1,\,0,\,0)$, $r=(0,\,0,\,-1)$, то можно определить
    начальную координату каждого маленького кубика в кубике Рубика как сумму
    цветов входящих в маленький кубик. Так можно было ввести кодировку
    состояния кубика Рубика, но она получится неполной. Чтобы убедиться в этом
    достаточно посмотреть на слово $\alpha = oywgbrowygbr$. Если его применить
    к начальному состоянию, вершинная кодировка покажет тривиальную
    перестановку, но мы не получим начальное состояние. Так происходит потому,
    что это слово меняет ориентацию некоторых кубов на границах двух цветов.
Слово $\alpha^2=e$ уже будет тривиальным.}

Для удобства мы будем обозначать элементы цветом и индексом, например, $b_5$
будет обозначать центральный синий стиркер. Индексы на каждой грани
определяются в соответствии с рисунком~\ref{fig:indexes}.

Как только мы ввели кодировку, можем проследить какой стикер стоит на какой позиции. И уже отсюда можем вытащить перестановки.


% \begin{marginfigure}
%     \includegraphics[width=1.1\columnwidth]{../pics/rubik_prepresentation.pdf}
%     \caption{Разрез кубика на две ленты с нумерациями.}
%     \label{fig:representation}
%     %\setfloatalignment{b}
% \end{marginfigure}

\subsection{Как так}

\end{document}
