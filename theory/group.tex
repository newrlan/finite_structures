% This is a modified version of the tufte-latex book example in which the title page and the contents page resemble Tufte's VDQI book, using Kevin Godby's code from this thread at https://groups.google.com/forum/#!topic/tufte-latex/ujdzrktC1BQ.
%
%% Unfortunately for the contents to contain
%% the "Parts" lines successfully, hyperref
%% needs to be disabled.
\documentclass{tufte-book}
% \usepackage[T2A]{fontenc} % Use 8-bit encoding that has 256 glyphs
\usepackage[OT1]{fontenc} % Use 8-bit encoding that has 256 glyphs
\usepackage[utf8]{inputenc} % Required for including letters with accents
\usepackage[russian]{babel}
% \usepackage{pscyr}
%   \renewcommand{\rmdefault}{ftm}
\title{Конечные структуры \\ {\Huge практические задачи}} % The article title

\author{Садыков Нурлан}

% Абстракт
% Заметки по решению кубика рубика с помощью теории групп.

\begin{document}

\frontmatter

\maketitle  
\section{Кубик Рубика}
\subsection{Где здесь группа?}

Обозначим вращения граней кубика\footnote{Под вращением грани понимаем ее
поворот на $90^\circ$ относительно остальной части кубика.} в соответствии с
цветом центрального стикера на грани. Это удобно потому, что вращения граней
всегда оставляют центральные стикеры на месте. Будем обозначать эти вращения
цветами соответствующих граней: $o$-оранжевый, $b$-синий, $r$-красный,
$y$-желтый, $w$-белый, $g$-зеленый. Вращения граней кубика рубика являются
образующими свободной группы. Любое слово в этой группе можно применить как
инструкцию к кубику рубика. Будем считать слово тривиальным, если после его
применения к собранному кубику, мы снова получим собранный
кубик.\footnote{Любые промежуточные состояния применения слова к кубику не
обязаны давать собранный вариант.} Например, очевидно, что $b^4 = bbbb = e$ то
есть вращение относительно синей грани четыре раза --- тривиальное слово,
поскольку снова дает правильную сборку.

Для определенности будем считать, что мы применяем слова к кубику слева
направо. То есть слово $oby$ это сперва повернуть оранжевую грань на
$90^\circ$, потом синюю и только потом желтую.

\end{document}
