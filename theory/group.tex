% This is a modified version of the tufte-latex book example in which the title page and the contents page resemble Tufte's VDQI book, using Kevin Godby's code from this thread at https://groups.google.com/forum/#!topic/tufte-latex/ujdzrktC1BQ.
%
%% Unfortunately for the contents to contain
%% the "Parts" lines successfully, hyperref
%% needs to be disabled.
\documentclass{tufte-book}
% \usepackage[T2A]{fontenc} % Use 8-bit encoding that has 256 glyphs
\usepackage[OT1]{fontenc} % Use 8-bit encoding that has 256 glyphs
\usepackage[utf8]{inputenc} % Required for including letters with accents
\usepackage[russian]{babel}
% \usepackage{pscyr}
%   \renewcommand{\rmdefault}{ftm}
\title{Заметки по теории групп {\Huge на примере кубика Рубика}} % The article title

\author{Садыков Нурлан}

% Абстракт
% Заметки по решению кубика рубика с помощью теории групп.

\begin{document}

\frontmatter

\maketitle  

\section{Где здесь группа}

% The Plan:
%   1. Цвета и грани
%   2. Элементарные преобразования - образующие
%   3. Инструкция - слово
%   4. Умножение состояний - группа

Кубик имеет шесть граней, которые в начальном состоянии раскрашены каждый в свой цвет. Мы будем использовать такие:
\begin{enumerate}
    \item[$O$] оранжевый
    \item[$B$] синий
    \item[$R$] красный
    \item[$Y$] желтый
    \item[$W$] белый
    \item[$G$] зеленый
\end{enumerate}

Если провернуть какую-либо грань, то мы осуществим преобразование над кубиком.
Заметим, что центральный стикер при повороте грани перейдет сам себя, то есть
останется на месте. Воспользуемся этим фактом, чтобы ввести кодировку
преобразований и обозначим такой поворот как $g_X$, здесь $X$ обозначает цвет
центрального стикера.\footnote{Можно ли ввести такое преобразование над
кубиком, чтобы cдвинуть все центральные стикеры?} Эти шесть движений, по одному
на каждый цвет, являются образующими нашей группы преобразований.

Для кубика рубика группа это не сам кубик, а множество преобразований которые
мы можем над этим кубиком сделать. 

Над кубиком Рубика можно сделать шесть элементарных преобразований --- это
вращения различных граней кубика. Очевидно, что любое состояние кубика можно получить как последовательное применение инструкции.

Причем центральный стикер во время вращения
останется на месте, то есть перейдет сам в себя. Этот факт окажется полезным
нам позже для задания кодировки.




Формальное определение группы таково:

Пусть~$G$ множество элементов, а $\circ$ - бинарная операция над~$G$. Пара $(G,\,
\circ)$ образует группу если

\begin{enumerate}
    \item Для любых трех элементов $a,b,c \in G$ выполняется ассоциативность $(ab)c = a(bc)$.
    \item Существует нейтральный элемент~$e$ такой, что для любого $g \in G$ выполняется $eg = ge = g$.
    \item Для любого элемента $g \in G$ существует обратный элемент $g^{-1} \in G$ такой, что $gg^{-1} = g^{-1}g = e$.
\end{enumerate}

\subsection{Несуществующая подгруппа}
$$ p = (1\,3\,6\,7) (9\,12\,16\,18\,17\,19\,13) (10\,14)
= (1\,3\,6\,7) ()
$$


\end{document}
